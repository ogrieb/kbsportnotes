\documentclass[a4paper,11pt,oneside]{article}
\usepackage{multirow}
\usepackage[colorlinks]{hyperref}

\author{Oliver Grieb}
\title{Notes on Kettlebell Sport}

\begin{document}
\maketitle

\begin{abstract}
This document gathers my notes and observations on Kettlebell Sport aka Girevoy 
Sport training.
\end{abstract}


\section{Assistance Training}


\subsection{Swings}

The \emph{swing} exercise in kettlebell sport is performed in several variants. 
Gireviks will often refer to different variants when they demonstrate a 'swing'.
This list gives an overview:

\begin{itemize}
  \item The most encountered GPP variant is the one or two armed \emph{(russian) swing}
  up to shoulder level, demonstrated \href{https://youtu.be/T2Sx7qi1TcQ?t=167}{here} 
  by Ruslan Rudnev~\cite{rkbyten_swings}. Watch the entire video for a presentation 
  of more fitness and GPP variants.
  
  \item Demonstrations of one armed swing SPP variants as snatch assistance training
  differ noticibly from the GPP variants by including the acceleration pull phase:
    \begin{itemize}
      \item Demonstrated  
      \href{https://youtu.be/T2Sx7qi1TcQ?t=167}{here} by Ruslan 
      Rudnev~\cite{rkbyten_swings},

      \item up to chest level \href{https://youtu.be/J3Vict1FzZI}{here} by 
      Sergey Merkulin~\cite{merkulin_swings}, and

      \item up to hip level \href{https://youtu.be/kOKHlc8EK8o}{here} by 
      Anton Anasenko~\cite{anasenko_swings}.
    \end{itemize}
  
  \item 
\end{itemize}

The one or two armed \emph{(russian) swing} with outstretched arms up to shoulder 
level is only trained in GPP complexes. See it demonstrated 
\href{https://youtu.be/T2Sx7qi1TcQ?t=167}{here} by Ruslan Rudnev~\cite{rkbyten_swings}. 
The upper part of the swing has the weight just travel by momentum, where in a 
snatch  or clean the acceleration pull would start. This just doesn't carry over 
well to the classic lifts. 

Swings are still programmed in SPP combinations with a snatch or long cycle, but 
mainly to reduce intensity of the main lift set. See for example the `black snatch' 
or `black long cycle' complexes described in section~\ref{blacksnatchlc}.

In these SPP combinations the weight is swung only up to the point of the acceleration 
pull. That starting point, respectively height, depends on a girevik's techniqe, i.e., 
some swing higher than others before they begin the acceleration pull.


\subsection{Swing with Acceleration Pull and High Pull}

\href{https://youtu.be/T2Sx7qi1TcQ?t=369}{accpull}


\subsection{Black Snatch and Black Long Cycle}
\label{blacksnatchlc}

\emph{Black Snatch} describes a swing+snatch complex usually done for time. Each 
snatch is paired with one or two extra swings. A typical set may look like:

\begin{center}
\begin{tabular}{c}
\hline
Snatch + Swing 3 x 2:00/2:00/-- \\
Snatch with extra Swing for 12 min, \\
switch hands every 2 min \\
\hline
\end{tabular}
\end{center}

Upper back, and grip endurance in particular, is strained by the acceleration pull 
in snatches. The extra swing brings down the snatch pace and the complex becomes 
less intense and thus more approachable for beginners.
For example, the 
\begin{itemize}
  \item \href{https://www.elitegirevoy.com/kettlebell-lifting-beginners-program-men/}
         {EGSA Kettlebell Lifting Beginner Program - Men}~
         \cite{egsa_kb_lifting_beginner_program_men} and
  \item \href{https://www.elitegirevoy.com/kettlebell-lifting-beginners-program-women/}
         {EGSA Kettlebell Lifting Beginner Program - Women}~
         \cite{egsa_kb_lifting_beginner_program_women}
\end{itemize}
program a progression of such snatch+swing complexes after their `static work' 
complexes. Progression is for more time per hand, as well as for more time snatched 
overall, while not extending beyond 6 minutes per hand.

A black snatch complex can be programmed for the following goals:

\begin{itemize}

  \item{Special Endurance:} The kettlebell is continuously held and swung. Thus, 
  special endurance for the backswing and upswing snatch phases, as well as grip 
  strength can be trained at less intensity and for a longer set. The swing is also 
  the classic assistance exercise for the snatch. Thus, the extra swing carries over 
  to improving general snatch endurance.
  
  \item{Weight Progression:} The assisting property of the swing can especially help 
  in beginner and intermediate programs to ease progression into snatching a higher 
  weight.
  
  \item{Technique Improvement:} The slower pace causes less mental strain and thus 
  enables gireviks to better concentrate and improve on particular aspects of their 
  form. The extra swing can for example be used to practice timing the acceleration 
  pull as proposed by S.~Rudnev~\cite{rkbyten_snatch_beginner_mistakes} in his video 
  \href{https://youtu.be/N186yUP9LcU?t=89}{Snatch: Mistakes of beginners}  

\end{itemize}



Performing the snatch in this 
complex with a `normal' one-breath-pause top fixation emphasizes training of special 
endurance for the backswing and upswing snatch phases, plus grip strength endurance in particular. 



The lessened intensity on the upper 
, giving the girevik several options:


\emph{Black Long Cycle}

\newpage
\begin{tabular}{clllllllll}

Cycle & Training 1 & Training 2 & Training 3 \\

\hline

\multirow{5}{*}{1} 	& A: Jerk 8 x 1:00/1:00 		& A: Complex 3 x 					& A: Complex over 6:00 \\
				 	& B: Snatch 4 x 1:00/1:00/-- 	& A1: 2 Swing + C\&J 3:00/2:00 		& A1: 0:10 Farmer Hold \\
  					&								& A2: Swing + C\&J 2:00/1:00		& A2: 0:10 Rack Hold \\
  					&								& A3: C\&J 1:00/3:00 				& A3: 0:10 Lockout Hold \\
  					&								&									& B: Snatch + Swing 2 x 2:00/2:00/-- \\

\hline

\multirow{5}{*}{2} 	& A: Jerk 10 x 1:00/1:00 		& A: Complex 2 x 					& A: Complex over 8:00 \\
					& B: Snatch 5 x 1:00/1:00/-- 	& A1: 2 Swing + C\&J 2:00/2:00 		& A1: 0:10 Farmer Hold \\
  					& 								& A2: Swing + C\&J 2:00/2:00		& A2: 0:10 Rack Hold \\
  					& 								& A3: C\&J 2:00/4:00 				& A3: 0:10 Lockout Hold \\
  					&								&									& B: Snatch + Swing 3 x 2:00/2:00/-- \\

\hline

\multirow{4}{*}{3} 	& A: Jerk 12 x 1:00/1:00 		& A: Complex \\
					& B: Snatch 6 x 1:00/1:00/-- 	& A1: 2 Swing + C\&J 3:00/3:00 \\
  					& 								& A2: Swing + C\&J 3:00/3:00 \\
  					& 								& A3: C\&J 3:00 \\

\hline
\end{tabular}

\bibliography{kbsport}
\bibliographystyle{unsrt}

\end{document}