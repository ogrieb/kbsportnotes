\documentclass[a4paper,11pt,oneside]{article}
\usepackage{multirow}
\usepackage[colorlinks]{hyperref}

\author{Oliver Grieb}
\title{Notes on Kettlebell Sport}

\begin{document}
\maketitle

\begin{abstract}
This document gathers my notes on Kettlebell Sport aka Girevoy Sport training.
\end{abstract}


\section{Assistance Training}


\subsection{Swings}

In this work we define a \emph{swing} as a pendulum swing with an outstretched arm 
to hip or chest height.
\emph{One armed swings} don't seem to be programmed in pure swing sets. The technique 
is easy and cardio intensity is low. The same observation can be made for 
\emph{two arm swings}.

One or two arm swings are mostly programmed in combinations, usually with another main 
exercise. This reduces the main exercise's intensity. See for example the `black snatch' 
or `black long cycle' complexes described in section~\ref{blacksnatchlc}.


\subsection{Swing with Acceleration Pull and High Pull}




\subsection{Black Snatch and Black Long Cycle}
\label{blacksnatchlc}

\emph{Black Snatch} describes a swing+snatch complex usually done for time. Each 
snatch is paired with one or two extra swings. A typical set may look like:

\begin{center}
\begin{tabular}{c}
\hline
Snatch + Swing 3 x 2:00/2:00/-- \\
Snatch with extra Swing for 12 min, \\
switch hands every 2 min \\
\hline
\end{tabular}
\end{center}

Upper back, and grip endurance in particular, is strained by the acceleration pull 
in snatches. The extra swing brings down the snatch pace and the complex becomes 
less intense and thus more approachable for beginners.
For example, the 
\begin{itemize}
  \item \href{https://www.elitegirevoy.com/kettlebell-lifting-beginners-program-men/}
         {EGSA Kettlebell Lifting Beginner Program - Men}~
         \cite{egsa_kb_lifting_beginner_program_men} and
  \item \href{https://www.elitegirevoy.com/kettlebell-lifting-beginners-program-women/}
         {EGSA Kettlebell Lifting Beginner Program - Women}~
         \cite{egsa_kb_lifting_beginner_program_women}
\end{itemize}
program a progression of such snatch+swing complexes after their `static work' 
complexes. Progression is for more time per hand, as well as for more time snatched 
overall, while not extending beyond 6 minutes per hand.

A black snatch complex can be programmed for the following goals:

\begin{itemize}

  \item{Special Endurance:} The kettlebell is continuously held and swung. Thus, 
  special endurance for the backswing and upswing snatch phases, as well as grip 
  strength can be trained at less intensity and for a longer set. The swing is also 
  the classic assistance exercise for the snatch. Thus, the extra swing carries over 
  to improving general snatch endurance.
  
  \item{Weight Progression:} The assisting property of the swing can especially help 
  in beginner and intermediate programs to ease progression into snatching a higher 
  weight.
  
  \item{Technique Improvement:} The slower pace causes less mental strain and thus 
  enables gireviks to better concentrate and improve on particular aspects of their 
  form. The extra swing can for example be used to practice timing the acceleration 
  pull as proposed by S.~Rudnev~\cite{rkbyten_snatch_beginner_mistakes} in his video 
  \href{https://youtu.be/N186yUP9LcU?t=89}{Snatch: Mistakes of beginners}  

\end{itemize}



Performing the snatch in this 
complex with a `normal' one-breath-pause top fixation emphasizes training of special 
endurance for the backswing and upswing snatch phases, plus grip strength endurance in particular. 



The lessened intensity on the upper 
, giving the girevik several options:


\emph{Black Long Cycle}

\newpage
\begin{tabular}{clllllllll}

Cycle & Training 1 & Training 2 & Training 3 \\

\hline

\multirow{5}{*}{1} 	& A: Jerk 8 x 1:00/1:00 		& A: Complex 3 x 					& A: Complex over 6:00 \\
				 	& B: Snatch 4 x 1:00/1:00/-- 	& A1: 2 Swing + C\&J 3:00/2:00 		& A1: 0:10 Farmer Hold \\
  					&								& A2: Swing + C\&J 2:00/1:00		& A2: 0:10 Rack Hold \\
  					&								& A3: C\&J 1:00/3:00 				& A3: 0:10 Lockout Hold \\
  					&								&									& B: Snatch + Swing 2 x 2:00/2:00/-- \\

\hline

\multirow{5}{*}{2} 	& A: Jerk 10 x 1:00/1:00 		& A: Complex 2 x 					& A: Complex over 8:00 \\
					& B: Snatch 5 x 1:00/1:00/-- 	& A1: 2 Swing + C\&J 2:00/2:00 		& A1: 0:10 Farmer Hold \\
  					& 								& A2: Swing + C\&J 2:00/2:00		& A2: 0:10 Rack Hold \\
  					& 								& A3: C\&J 2:00/4:00 				& A3: 0:10 Lockout Hold \\
  					&								&									& B: Snatch + Swing 3 x 2:00/2:00/-- \\

\hline

\multirow{4}{*}{3} 	& A: Jerk 12 x 1:00/1:00 		& A: Complex \\
					& B: Snatch 6 x 1:00/1:00/-- 	& A1: 2 Swing + C\&J 3:00/3:00 \\
  					& 								& A2: Swing + C\&J 3:00/3:00 \\
  					& 								& A3: C\&J 3:00 \\

\hline
\end{tabular}

\bibliography{kbsport}
\bibliographystyle{siam}

\end{document}